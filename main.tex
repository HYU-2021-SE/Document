
% VLDB template version of 2020-08-03 enhances the ACM template, version 1.7.0:
% https://www.acm.org/publications/proceedings-template
% The ACM Latex guide provides further information about the ACM template

\documentclass[sigconf, nonacm]{acmart}
\usepackage{booktabs}

%% The following content must be adapted for the final version
% paper-specific
\newcommand\vldbdoi{XX.XX/XXX.XX}
\newcommand\vldbpages{XXX-XXX}
% issue-specific
\newcommand\vldbvolume{14}
\newcommand\vldbissue{1}
\newcommand\vldbyear{2020}
% should be fine as it is
\newcommand\vldbauthors{\authors}
\newcommand\vldbtitle{\shorttitle} 
% leave empty if no availability url should be set
\newcommand\vldbavailabilityurl{URL_TO_YOUR_ARTIFACTS}
% whether page numbers should be shown or not, use 'plain' for review versions, 'empty' for camera ready
\newcommand\vldbpagestyle{plain} 

% Some/most Computer Society conferences require the compsoc mode option,
% but others may want the standard conference format.
%
% If IEEEtran.cls has not been installed into the LaTeX system files,
% manually specify the path to it like:
% \documentclass[conference,compsoc]{../sty/IEEEtran}

% 직접 추가한 패키지
\usepackage{cleveref, array, booktabs, threeparttable}
\usepackage[labelsep=period, font={footnotesize, sc}]{caption}

\usepackage{tabularx}
\usepackage{graphicx}
% 구역 주석
\usepackage{comment}
\usepackage{kotex}
\usepackage{makecell}

\begin{document}
\title{Smart Wine Cellar – Cave de Vin(Label D)\\DIOnyoS}

%%
%% The "author" command and its associated commands are used to define the authors and their affiliations.
\author{Yujin Do}
\affiliation{%
  \institution{College of Engineering\\Hanyang Unviersity\\Dept. of Information System}
  \city{Seoul, Korea}
}
\email{yujindo9@gmail.com}

\author{Yerin Cho}
\affiliation{%
  \institution{College of Engineering\\Hanyang Unviersity\\Dept. of Information System}
  \city{Seoul, Korea}
}
\email{forjustice9839@gmail.com}

\author{Jungi Kim}
\affiliation{%
  \institution{College of Engineering\\Hanyang Unviersity\\Dept. of Information System}
  \city{Seoul, Korea}
}
\email{damnum3@hanyang.ac.kr}

\author{Hyungseo Han}
\affiliation{%
  \institution{College of Engineering\\Hanyang Unviersity\\Dept. of Information System}
  \city{Seoul, Korea}
}
\email{shorelinesquere@gmail.com}





%%
%% The abstract is a short summary of the work to be presented in the
%% article.
\begin{abstract}
Our team is trying to develop User-Friendly wine cellar by appending additional functions to “LG wine cellar". By capturing and registering wines in "Cave de Vin(label D)", users can effectively manage their wines and wine cellar. Using wine recommendation and purchasing function can promote additional consumptions to wine and wine cellar. Also, by sharing the scene that enjoying wine on the social media, people can enjoy the marketing effect of LG wine cellar.
\end{abstract}

%%%for splitting
\maketitle

%%% Table 1 : Role Assignment
\begin{table}[h]
\caption{Role Assignment}
\def\arraystretch{1.24} \small
\begin{tabular}{|p{1.4cm}|p{1.4cm}|p{4.4cm}|}
\hline
\textbf{Roles} & \textbf{Name} & \textbf{Task description and etc.} \\ \hline
Quality \par Assurance \par Manager&  Hyungseo Han & He understand demand and requirement of application. They test the overall application from customers’ point of view and report error and bug. They strive to maximize user experience by considering UI / UX. \\ \hline

Software \par Developer & Yerin Cho \par Jungi Kim  & Software developers think about the software in general. They design and implement the requirements of application. They also strive to satisfy users and customers. \\ \hline

Development \par manager & Yujin Do & Development manager manages the overall project. She strive to meet the needs of customers. Also, she collects the feedback from users. After collecting feedback, she advises software developers to improve application quality. \\ \hline
\end{tabular}
\end{table}

\section{Introduction}\noindent\subsection{Motivation}
\indent Due to the COVID-19, the number of people who drink at home has increased significantly. Wine is one of the most popular drinks for people who drink alone, and wine imports rose 110\% year-on-year in the first half of last year. Thus, the demand for wine cellars also increased. However, there are too much information to enjoy wine, such as the way to store wine, or detailed information of the wines. But services that support people who try to get information of in the wine in Korea are poor. Therefore, our team felt the needs for a service that effectively records and manages the wine that I drink and drank, and further recommends and induces to user to purchase the wine.\\
 
\noindent\subsection{Problem Statement}
\indent There are many wine applications such as Vivino. However, there is a limitation that the service is based in a foreign country other than Korea. Currently, it is difficult to find information to better enjoy wine in Korea. Accordingly, the necessity of providing various wine information, wine purchase service, SNS sharing service, etc. to enjoy wine better by linking with the wine cellar to facilitate humidity and temperature control was recognized.\\ 

\noindent\subsection{Solution}
\begin{enumerate}
\item Available to manage wine and wine cellar effectively.
\begin{itemize}
\item It is possible to manage users' wines and wine cellar in one
application.
\item Effective wine management is possible by recommending
appropriate wine storage methods to users.
\item Once the wine is registered, detailed information can be
easily offered.
\end{itemize}
\item Available to create additional consumption of wine cellar. 
\begin{itemize}
\item By recommending new wines based on registered wines,
new wines can be made, and even wine cellars can be
consumed. B2B business development through collaboration
with wine sellers is also possible.
\item By providing a user-friendly SNS sharing service, the
promotional effect of LG Wine Cellar can also be expected.
\end{itemize}
\end{enumerate}

\noindent \subsection{Related Software}
\begin{enumerate}
    \item {\textbf{Vivino}}\\
    Vivino is the world’s largest wine application, which helps users discover the perfect wine. When users scan wine label, it shows ratings, reviews, and prices of the wine. Users can easily click to purchase bottles on web or mobile in application. And it keeps track of wines which users like and recommend other wine for user.\\
    \item {\textbf{My Own Refrigerator (from GS25 convenience store)}}\\
    My Own Refrigerator is a convenience store shopping service that allows user to discount, earn, pay directly from mobile. And user can store products purchased at the store. One of key features is ‘Storage Box’. Users can check the storage history of purchased products and can use them later.\\
    \item {\textbf{CellWine}}\\
    CellWine is an application which provides mobile wine collection. When users scan wine labels, application keeps track of cellars and log the cost and quantity of collections anytime, anywhere. And it helps users to record own rating score and flavors. It also helps users to explore different wine by giving recommendations.\\
    \item {\textbf{ThinQ}}\\
    ThinQ is an application which connects washer, air conditioner, TV and other LG appliances and mobile phones. It automatically accesses the current state of products and get immediate notifications of products. Existing LG wine cellar only supports few functions. Users can control temperature of wine upper, wine lower and fridge drawer by ThinQ. And if users say, ‘Hi LG, open the refrigerator door’, or put your foot close to the product, the door open automatically.
\end{enumerate}
%%% End of Section 1 Introduction

%%% Beginning of Section 2 
\section{Requirements}

\noindent\subsection{Wine cellar registration screen}

\begin{enumerate}
\item Select the LG wine cellar that the customer purchased.
\item Enter and register the serial number of the purchased \\product.
\item After authenticating the information (serial number, model), the application automatically creates the shape of wine cellar that user purchased and an empty floor of wine cellar.
\end{enumerate}

%%%Beginning of Section 2.2
\subsection{Wine cellar screen - default screen after registered model}
\begin{enumerate}
\item \textbf{Floors management menu}

\begin{enumerate}
\item Wine add – on
 \begin{enumerate}
    \item Customers take a photograph of the label of the wine
    \item The application analyzes the label that from user-taken
	photograph.
	\begin{enumerate}
	\item Application analyzes wine name, and date of 
    manufacture.
    \item Customers enter the purchase date.
    \item Application shows information related to the 
    registered wine - tasting notes, average price, country,
    grape variety, food pairing, winery, vintage,
    comparison, management method, recommended
    glass, etc.
	\end{enumerate}
	\item After judging whether the temperature and humidity of the floor selected by the user is suitable, the application
can recommend another floor
\end{enumerate}

\item Temperature, Humidity management of each floor of model
\begin{enumerate}
    \item Inside the app, the temperature of each floor and the humidity of the model can be monitored on the left or right side of the virtual model.
    \item If the customers want to adjust the temperature and 	humidity, then customers can modify these by pressing 	the mark in the application.
\end{enumerate}

\item  Wine cellar lock function
\begin{enumerate}
    \item Customers can activate lock function to full floor or 	floor by-floor via application.
    \item To prevent theft and damage to wine cellar, customers 	can set alarms on locked floors.
\end{enumerate}

\end{enumerate}

\item \textbf{Individual wine screen}
\begin{enumerate}
\item Detailed information about selected wine
\begin{enumerate}
    \item Customers choose one of the wines in their own wine
    cellar model.
    \item The application displays detailed information about 	the selected wine, such as manufacture date, expiration 	date, purchase date, alcohol percentage, etc.
    \item When customers drink a bottle of wine, user can delete the wine from wine cellar by clicking 'empty?' button.\\
\end{enumerate}
\end{enumerate}
\item \textbf{Wine history screen}
\begin{enumerate}
\item Show the whole of wines that customer has drank so far.
\begin{enumerate}
    \item Display the drank wines using wine corks.
    \item Recommends of wines that are likely to suit
	customers’ taste based on the history of wine customers	have drunk.
	\item Recommendations are based on price, grape variety,  
   	taste and smell, and customers’ nationality.
\end{enumerate}
\end{enumerate}
\end{enumerate}

%%% End of subsection 2.2
%%% Beginning of subsection 2.3
\subsection{Social media quick link function}
\begin{enumerate}
    \item \textbf{This function helps user to share wine and wine cellar which he
has on social media such as ‘Instagram’, ‘Twitter’, ‘Facebook’
and so on.\\}

\item \textbf{Auto complete hashtag}

When user uses ‘social media quick link’, user can get ‘auto
complete hashtag’ function. Hashtag is made based on the
name of wine cellar, and the name, type, taste of the wine,
which helps users upload the post easily on social media.\\

\item \textbf{Share my Wine Cellar}
  User can share the image of wine cellar. By capturing the
image of virtual wine cellar, user can save the image into jpg
file format or share it on social media.\\

\item \textbf{Share my Wine History}
\begin{enumerate}
    \item User can share ‘my wine history’
    \item The wine he has drunk can be shown into the cork icon.
The more he drinks wine, the more cork stamps will be
collected.
    \item The price of wine is shown in the image of receipt, which
shows each name of wines and each price and total price.\\
\end{enumerate}

\item \textbf{Make my Wine Topster\\}
User can make ‘wine topster’, which means collage of
favorite wine.
\begin{enumerate}
    \item First, user sets the layout of topster. User can set the
rows, columns (n*n) and background image of collage.
\item Second, user makes the collage based on the images
of wine label which were saved in application before.
\item Third, user can download wine topster into jpg/png
file format or can share it on social media directly.
\end{enumerate}
\end{enumerate}
%%%End of subsection 2.3
%%%Beginning of subsection 2.4
\subsection{Wine Recommendation function}
\begin{enumerate}
\item \textbf{Wine promotion nearby to me}
When there are discounts on wine near to user, application gives push alarm to user. It helps user to make a reservation or to buy wine at reasonable price.\\

\item \textbf{Wine Reservation}
If user is looking for rare wine, user can register wine alarm which relates to wine shop. When wine is stocked, alarm goes off and lets user know that wine is available to buy. Then user can make a reservation or buy wine.\\

\item \textbf{Calendar Sync}
When user synchronizes google calendar with application, it recommends wine which fits in with schedule. Recommendation can be made based on wine which user already has. If there isn't any suitable wine, application recommends user to purchase wine from wine shop nearby.\\
\end{enumerate}
%%%End of subsection 2.4 & section 2
%%%Beginning of section 3
\section{Development Environment}
\noindent \subsection{Choice of software development platform}

\indent Our team will develop application in the environment of Mac operating system. And our application will run on the iOS environment. We'll use the React-Native framework that can develop Android and IOS applications simultaneously. Therefore, it'll make it easier to develop Android OS applications in the future. Therefore, our team will use to create applications based on the ios platform, and usage of the languages will be React-Native, Java Spring, and MariaDB.\\ \\

\begin{enumerate}
    \item \textbf{React-Native\\}
    React Native is an open source mobile application framework developed by Facebook. It is used to develop applications for Android, iOS, Web, and UWP, allowing developers to use React in addition to native platform features. React Native has the biggest advantage of being able to create native UI for Android and iOS using JavaScript, and create high-quality UI faster than using XML/Java/Kotlin. In addition, we can quickly check changes or modifications with our eyes, making QA or feedback easier to proceed. \\
    
    \item \textbf{Spring - JAVA Framework\\}
    The Spring Framework is an open source application framework for the Java platform. It is used as a foundation technology for the e-government standard framework recommended for use in the development of web services by public institutions in Korea.\\
    
    \item \textbf{MariaDB\\}
    It is necessary to store and manage information about wine cellars, users of wine cellars, wine stored in wine cellars. For this, MariaDB is used. MariaDB is an open-source relational database management system (RDBMS). It is based on the same source code as for MySQL and follows the GPL v2 license. It also supports the latest SQL functions such as common table expressions (CTE), window functions, temporary data tables, and JSON functions.
\end{enumerate}
%%%End of subsection 3.1
%%%Begin of subsection 3.2
\subsection{Software in use}
There are already two existing popular apps that manage and record purchased wine. Vivino was launched in 2010 and has recorded more than 50 million downloads so far and maintains a 4.6 rating in the Google Play Store. As it has been a long time since it was released, it has many users' evaluations and reviews.\\
Cellwine was launched in 2016, recording more than 50,000 downloads, and has a rating of 4.0 points in the Play Store. Both Vivino and CellWine can register wine by taking wine labels and record wine that users have been drinking so far. And users can search for a variety of wines around the world, check the ratings and information of each wine, reviews from other tasters, prices, and recommended foods that goes well with wine. \\
Based on the wines that have been drunk and recorded so far, those two apps also recommend new wines that suit the user. Vivino also has the function to purchase wine from the application.\\
%%%End of subsection 3.2
%%%Begin of subsection 3.3
\subsection{Task Distribution}
\begin{table}[hb]% h asks to places the floating element [h]ere.
  \caption{Task Distribution}
  \label{tab:freq}
  \begin{tabular}{|p{3cm}|p{4.4cm}|}
    \toprule
    \textbf{Name} & \textbf{Task}\\
    \midrule
    Yerin Cho & App backend \\
    Yujin Do & App frontend, UI/UX \\
    Hyungseo Han & App backend, App frontend \\
    Jungi Kim & App frontend, Infra \\
  \bottomrule
\end{tabular}
\end{table}

\section{Specification}
%%%End of subsection 3.3 & section 3
%%%Beginning of section 4

%%%Beginning of subsection 4.1
\noindent \subsection{Page}
Welcome page(\textbf{\autoref{fig:Initial}}) is the first screen when execute this application. If user runs the application, welcome page, which is consisted of ‘Wine cellar registration’ button and ‘go to My Wine cellar’ button.(\textbf{\autoref{fig:Guidance}})
\begin{figure}
  \centering
  \includegraphics[width=4cm]{firstpage.png}
  \caption{Welcome Page}
  \label{fig:Initial}
\end{figure}

\begin{figure}
  \centering
  \includegraphics[width=4cm, height=6cm]{guidance.png}
  \caption{Guidance Page}
  \label{fig:Guidance}
\end{figure}
%%%End of subsection 4.1
%%%End of section 4


\noindent \subsection{Wine cellar registration}
%%%\begin{table}[hb]% h asks to places the floating element [h]ere.
  %%%\caption{Frequency of Special Characters}
  %%%\label{tab:freq}
 %%% \begin{tabular}{ccl}
    %%%\toprule
    %%%Non-English or Math & Frequency & %%%Comments\\
    %%%\midrule
    %%%\O & 1 in 1000& For Swedish names\\
    %%%$\pi$ & 1 in 5 & Common in math\\
    %%%\$ & 4 in 5 & Used in business\\
    %%%$\Psi^2_1$ & 1 in 40\,000 & %%%Unexplained usage\\
  %%%\bottomrule
%%%\end{tabular}
%%%\end{table}
\begin{figure}
  \centering
  \includegraphics[width=4cm, height=6cm]{winecellarregi.png}
  \caption{Wine cellar Registration Page}
  \label{fig:Wine cellar Registration Page}
\end{figure}

\begin{figure}
  \centering
  \includegraphics[width=4cm, height=6cm]{regicheck.png}
  \caption{Wine cellar Registration Check Page}
  \label{fig:Wine cellar Registration Check Page}
\end{figure}

Wine cellar registration button is for user who is first in
‘DIOnyoS’ application and who wants to register another
wine cellar in ‘DIOnyoS’. When user press this button,
user has to select the wine cellar which he purchased
from horizontal scroll. And then user has to enter the
serial number of the wine cellar. \textbf{(Figure 3)}\\
After user ends two steps for wine cellar registration,
there will be a page for check. If there isn’t any error in
registered information, user will push ‘ok’ button to go
to ‘My Wine cellar’, which is main page. Or else, user
will push ‘try again’ button to re-enter information for
wine cellar registration. \textbf{(Figure 4)}\\

\clearpage

\noindent \subsection{My Winecellar}
\begin{figure}
  \centering
  \includegraphics[width=4cm, height=6cm]{mywinece.png}
  \caption{My Winecellar: Main Page}
  \label{fig:My Winecellar}
\end{figure}

If user presses this button, user can go to ‘My Wine
cellar’, which is main page of application.
\textbf{(Figure 5)}\\

My Wine cellar is the main page of application. My Wine
cellar is virtual wine cellar, which shows image of registered
wine label in application. Wine cellar is consisted of 4 floors.
As there is difference in the size of wine cellar, main page
lets user to see wine by scrolling rows.\\
Navigation bar is consisted of 4 icons, \textit{plus}, \textit{lock}, \textit{alarm}
and \textit{share}.\\

\noindent \subsection{Wine registration}
When user presses \textbf{Plus} icon,
user starts wine registration step.\\

\begin{enumerate}


    \item User has to open camera application. After
taking photo of wine label, user needs to
determine whether to use the photo. If user
presses ‘try again’ button, user goes to camera
application again. Or else, user presses ‘use’
button, user starts wine registration step.
\textbf{(Figure 6)} \textbf{(Figure 7)}
\item User starts wine registration step by step.
Name of wine will be automatically completed
by getting name from scanned-wine label
database. But user needs to save vintage and
date of purchase. Also, user needs to save
location of wine, such as wine upper, wine lower. \textbf{(Figure 8)}
\item When user saves these information, pop up window
may appear to user. If user saves wine in appropriate floor. \textbf{(Figure 11)}
\item Based on the information, wine information page is created. \textbf{(Figure 9, Figure 10)}
\end{enumerate}

\begin{figure}
  \centering
  \includegraphics[width=4cm, height=6cm]{camera.png}
  \caption{Camera roll 1}
  \label{fig:My Winecellar}
\end{figure}

\begin{figure}
  \centering
  \includegraphics[width=4cm, height=6cm]{photo.png}
  \caption{Camera roll 2}
  \label{fig:My Winecellar}
\end{figure}

\begin{figure}
  \centering
  \includegraphics[width=4cm, height=6cm]{regiinfo.png}
  \caption{Wine registration}
  \label{fig:My Winecellar}
\end{figure}

\clearpage

\begin{figure}
  \centering
  \includegraphics[width=4cm, height=6cm]{wineinfo.png}
  \caption{Wine Information}
  \label{fig:My Winecellar}
\end{figure}

\begin{figure}
  \centering
  \includegraphics[width=4cm, height=6cm]{wineinfo2.png}
  \caption{Wine Information}
  \label{fig:My Winecellar}
\end{figure}

\begin{figure}
  \centering
  \includegraphics[width=4cm, height=6cm]{infocheck.png}
  \caption{Floor Recommendation}
  \label{fig:My Winecellar}
\end{figure}

\noindent \subsection{Wine cellar Lock}
\begin{figure}
  \centering
  \includegraphics[width=4cm, height=6cm]{2. WineCellarPasswdSetting.png}
  \caption{Wine cellar Password Setting}
  \label{fig:wine cellar lock setup}
\end{figure}

\begin{figure}
  \centering
  \includegraphics[width=4cm, height=6cm]{4. Passwd Confirmation.png}
  \caption{Enter the Wine cellar Password}
  \label{fig:wine cellar password }
\end{figure}
\clearpage

\begin{figure}
  \centering
  \includegraphics[width=4cm, height=6cm]{5. ViewingCellarTrytoLock.png}
  \caption{Wine cellar Lock}
  \label{fig:wine cellar lock}
\end{figure}

\begin{enumerate}
    \item When user presses \textbf{lock} button, user will encounter Password Setting Page. User must set the password in 6 – digit number. If the password which user sets doesn’t meet the requirement, warning alert will pop-up. After user saves the password, user will go to the Wine Cellar Lock Main Page.  \textbf{(Figure 12)}
    \item To Lock the wine cellar, user must enter the 6-digit password first. If user enters the correct password, saved password will pop-up to let the user to know the saved password. Else if, page which lets user to enter the correct password will pop-up.  \textbf{(Figure 13)}
    \item After entering the correct password, user can lock the winecellar.  If user wants to lock full floor, user can click
‘lock full floor button’. Or else, user can lock floor by
floor by clicking each lock button. \textbf{(Figure 14)}
\end{enumerate}





\noindent \subsection{Wine Recommendation}

On the wine recommendation page, users can go to
pages that check nearby wine-related events (e.g.,
discounts), get information about ’nearby wine shops’,
’buy wine or make reservations’, and ’recommend
wine which fits in with their schedules’. \textbf{(Figure 15)}

\begin{figure}
  \centering
  \includegraphics[width=4cm, height=6cm]{winerec.png}
  \caption{Wine Recommendation}
  \label{fig:wine recommendation}
\end{figure}

\begin{figure}
  \centering
  \includegraphics[width=4cm, height=6cm]{winepromo.png}
  \caption{Wine Promotion}
  \label{fig:wine promotion}
\end{figure}


\begin{figure}
  \centering
  \includegraphics[width=4cm, height=6cm]{winepromoalarm.png}
  \caption{Wine Promotion Alarm}
  \label{fig:wine promotion alarm}
\end{figure}

\begin{enumerate}
    \item \textbf{Wine promotion nearby to me}
    \begin{enumerate}
        \item The application requests the user’s location
information. If the user allows the request, the
map displays wine promotion around the user.
\item The ’wine shop nearby to me’ view under
the ’map’ view displays wine shops around
the user. Users can set alarms for each
wine shop by clicking the bell icon on
the top right. If an event or promotion is
held at a wine shop with an alarm set, the
application sends a notification to the user.\textbf{(Figure 16)}
\item When the user clicks the notification, wines
corresponding to the event at the wine shop are
displayed. The app shows the name, price, and
testing notes of the wine.\textbf{(Figure 17)}
    \end{enumerate}

    \item \textbf{Wine reservation}\\
    
   
    On the Wine Reservation page,
users can search for and reserve the wine they want
to purchase.
 
   \begin{figure}
  \centering
  \includegraphics[width=4cm, height=6cm]{wineres.png}
  \caption{Wine Reservation}
  \label{fig:wine reservation}
\end{figure}

\begin{figure}
  \centering
  \includegraphics[width=4cm, height=6cm]{winearrive.png}
  \caption{Wine Arrival}
  \label{fig:wine arrival}
\end{figure}
\begin{enumerate}


    \item The user can search for wine by entering
the name of the wine. When the user writes
the name of the wine and presses the enter
button, the app shows several shops where
wine entered by the user can be purchased.\\
The user can press the bell button on the upper
right in each wine shop view to receive a
notification whenever the wine he or she wants
becomes available for purchase. App sends no-
tification if the wine user wants has arrived at
the wine shop. \textbf{(Figure 18)}
\item When wine that user wants arrives at the wine
shop where the user presses the bell button
and becomes available, the app sends a
notification to the user. When the user clicks
the notification, it goes to the Congratulation
page. On the Congratulation page, users can
see the name of the wine and the store they
reserved, and the question of whether to
purchase it ’right now’. \textbf{(Figure 19)}
\end{enumerate}
   
    \item \textbf{Calendar synchronize}

    \begin{enumerate}
        \item By logging in with their Google account, users can
link this app with their Google Calendar. \textbf{(Figure 20)}
        \item Using Google-Calendar-api, the application
displays the calendar of current month. When
the user clicks on the day with schedule, the
schedule of the day is displayed in the view under
the calendar. The daily schedule indicates the
date and time, the purpose of the schedule (e.g.,
evening, party), and who users are meeting with. \textbf{(Figure 21)}

        \item The app focuses on ’people’ who are together that
day. The recommended wine depends on who the
users are with that day. People who share schedule
with users are highlighted with white lines. If
the user presses the highlighted part (button), the
app goes to the ’wine information’ page of the
recommended wine. \textbf{(Figure 22)}
    \end{enumerate}
  
 \begin{figure}
  \centering
  \includegraphics[width=4cm, height=6cm]{calendar.png}
  \caption{Google calendar Sign In}
  \label{fig:google calendar sign in}
\end{figure}

\begin{figure}
  \centering
  \includegraphics[width=4cm, height=6cm]{calendar2.png}
  \caption{Google calendar synchronization}
  \label{fig:google calendar}
\end{figure}

\end{enumerate}



\clearpage

\subsection{Share}
Share, which is consisted of ‘share my Wine cellar’,
‘share my Wine history’, and ‘make my Wine Topster’,
lets user to share wine-related images in Instagram,
Facebook, Twitter and saves images to gallery in
common. When user shares these images, hashtag
\#LGwinecellar, \#MyWineCellar, and \#DIOnyoS will
be automatically completed. And it can lead to
promotion of wine cellar and application DIOnyoS.
\begin{enumerate}
    \item \textbf{Share my Wine Cellar}\\
    \begin{figure}
  \centering
  \includegraphics[width=4cm, height=6cm]{sharecel.png}
  \caption{Share My Winecellar}
  \label{fig:wine arrival}
\end{figure}
Share my Wine Cellar lets user to take screentshot
of ‘My WineCellar’, which is main page. \textbf{(Figure 21)}
    \item \textbf{Share my Wine History : Cork}\\
     \begin{figure}
  \centering
  \includegraphics[width=4cm, height=6cm]{sharecork.png}
  \caption{Share My Cork}
  \label{fig:wine arrival}
\end{figure}
    The number of wine which user has drunk
until now appears as the images of cork. And
it will look like loyalty card. So, the more user
drink, the more cork user can collect. 
\textbf{(Figure 22)}
    \item \textbf{Share my Wine History : Receipt}\\
     \begin{figure}
  \centering
  \includegraphics[width=4cm, height=6cm]{sharerec.png}
  \caption{Share My Receipt}
  \label{fig:wine arrival}
\end{figure}
    Wine which user has drunk until now appears
as the format of receipt. Receipt is consisted
of name, number and price of each wine. And
total amount and price of wine will appear at
the bottom of receipt. 
\textbf{(Figure 23)}
    \item \textbf{Make my Wine Topster}\\
     \begin{figure}
  \centering
  \includegraphics[width=4cm, height=6cm]{sharetop.png}
  \caption{Wine Topster Description}
  \label{fig:wine topster 0}
\end{figure}
\begin{figure}
  \centering
  \includegraphics[width=4cm, height=6cm]{top1.png}
  \caption{Wine Topster Step 1}
  \label{fig:wine topster 1}
\end{figure}
\begin{figure}
  \centering
  \includegraphics[width=4cm, height=6cm]{top2.png}
  \caption{Wine Topster Step 2}
  \label{fig:wine topster 2}
\end{figure}
\begin{enumerate}
    \item Description of making topster will
appear. When user clicks ‘Go’ button,
user can start making his own topster. \textbf{(Figure 24)}
\item User can set rows \& columns and
background image of wine topster.
\textbf{(Figure 25)}
\item User can make wine topster from setting. User
can retrieve image of wine label from database. \textbf{(Figure 26)}
\end{enumerate}
   

\end{enumerate}

\subsection{Wine information}
\begin{figure}
  \centering
  \includegraphics[width=4cm, height=6cm]{wineinfo.png}
  \caption{Wine Information page}
  \label{fig:wine info}
\end{figure}
\begin{figure}
  \centering
  \includegraphics[width=4cm, height=6cm]{wineinfo2.png}
  \caption{Wine Information page}
  \label{fig:wine info}
\end{figure}
All information appears in each container tag. Photo
of label, name, wine and date of purchase of wine
appear in upper tag. And summary of wine, such as
region, grapes and average price of wine appear in
middle tag. And information of wine, such as tasting
notes, food pairing, management and recommendation
glasses appear in lower tag.
\subsection{My WineCellar Settings}
Settings icon locates next to My Wine Cellar(main page).
If user pushes settings icon, user can control wine cellar.
\begin{figure}
  \centering
  \includegraphics[width=4cm, height=6cm]{setting.png}
  \caption{My Winecellar Settings}
  \label{fig:wine cellar settings}
\end{figure}
\begin{enumerate}
    \item Nickname\\
    The default name of wine cellar is ‘My Wine cel
lar–1’. User can change it and generate own nickname.
Nickname allows user who has many wine cellars to
distinguish one from many wine cellar.
    \item Temperature control\\
    A there is difference in appropriate temperature of
wine, user can control temperature of each floor of
wine cellar.
    \item Humidity control\\
    User can control humidity of wine cellar.
    \item Image of wine label\\
    When user clicks the image of wine label in main page,
user goes to wine information page directly.
\end{enumerate}


%%%\begin{table}[hb]
%%%  \caption{A double column table.}
%%%  \label{tab:freq}
%%%  \begin{tabular}{ccl}
%%%    \toprule
%%%    A Wide Command Column & A Random Number & Comments\\
%%%    \midrule
%%%    \verb|\tabular| & 100& The content of a table \\
%%%    \verb|\table|  & 300 & For floating tables \par within a single column\\
%%%    \verb|\table*| & 400 & For wider floating tables that span two columns\\
 %%%   \bottomrule
%%%  \end{tabular}
%%%\end{table}

\section{Architecture design and implementation}
\subsection{Overall Architecture}
\begin{figure}
  \centering
  \includegraphics[width=8cm, height=5cm]{overallarch.png}
  \caption{Overall Architecture}
  \label{fig:overall arch}
\end{figure}
For our DIOnyoS application, we have several modules
for the whole architecture. First, we used React-Native
for client side. It allows the application to perform several
functions and implement designs. Using the application,
users can register wines in winecellars and easily lock
the winecellars. Also, they can share the informations of
wine cellars in their social media.\\
 Second, we used Amazon AWS EC2, S3 for deploy-
ment, Spring framework for whole back-end section to
receive and serve data to front-end section as REST API,
and Node.js for running React-Native framework. Entire
API server except acquiring wine image file is running
on Amazon Web Services EC2 instance, and API server
for acquiring wine image file is running on Amazon
Web Services S3 instance. Below figure shows overall
architecture of the service.These modules are interacting
each other in the application
\subsection{Database Design}
\begin{figure}
  \centering
  \includegraphics[width=8cm, height=5cm]{database.png}
  \caption{Database Design}
  \label{fig:database design}
\end{figure}
\subsubsection{Member Table}
 \begin{enumerate}
        \item memberId(PK)\\
        identification of user
        \item createdAt\\
        date that user account created
        \item email\\
        user email
        \item memberName\\
        name of user
        \item updatedAt\\
        the date when user lastly modified their information
    \end{enumerate}

\subsubsection{WineCellar Table}
    \begin{enumerate}
        \item     winecellarId(PK)\\
        identification of wine cellar
        \item createdAt\\
        date that wine cellar registered to database
        \item humidity\\
        the information of humidity that applied at wine cellar
        \item lightColor\\
        the information of color that currently applied to wine cellar
        \item lock\\
        status of wine cellar whether locked or not
        \item lockPassword\\
        the password to lock/unlock wine cellar
        \item nickName\\
        unique Name of wine cellar that decided by user
        \item temperature\\
        the information of temperature that currently applied to wine cellar
        \item type\\
        the type of wine cellar
        \item updatedAt\\
        the date when user lastly modified information of their wine cellar
    \end{enumerate}
    \subsubsection{Wine Table}
     \begin{enumerate}
        \item wineId(PK)\\
        identification of wine
        \item winecellarId(FK)\\
        identification of wine cellar(wine cellar that storing wine)
        \item corkImage\\
        cork image of wine took by user
        \item createAt\\
        the date that wine registered to wine cellar
        \item labelImage\\
        label image of wine
        \item location\\
        the location of wine where it stored in wine cellar
        \item producedDate\\
        the date wine produced by manufacturer
        \item purchasedDate\\
        the date wine purchased by user
        \item updateAt\\ the date when user lastly modified information of their wine
        \item wineName\\ the name of wine
    \end{enumerate}

    \subsection{Directory organization}
    We are using 3 big modules which follow the overall architecture. First one is React Native, second one is Spring and the last one is Maria DB. Also, we have additional directory for LaTeX. 
    \subsubsection{\textbf{Server}}
    \begin{enumerate}
    \item Server/gradle/wrapper/ \\
    This directory is a collection of services that define what logic to deal with when building the application.
    \item Server/src/main/java/com/dionysos/winecellar/config \\
    This directory is a collection of files to facilitate server request processing. When user initialize the application, then WebConfig.java runs automatically to prepare user’s authentication requests.
    \item Server/src/main/java/com/dionysos/winecellar/domain/auth\\
    This directory is a collection of directories or files to handle user’s authentication requests. This directory is focused on handling authentication or confirmation of user’s token.
    \item Server/src/main/java/com/dionysos/winecellar/domain/member\\
    This directory is a collection of files to handle user request about member(user). ‘Config’ directory discern which requests are sent from client. After analyzation, ‘Service’ directory handles that request by logics that written in files in ‘Service’ directory. If SQL query requires, then files in ‘Dao’ directory create SQL query and send to database.
    \item Server/src/main/java/com/dionysos/winecellar/domain/wine\\
    This directory is a collection of files to handle user requests about wine. First, files from ‘Api’ directory receives the requests from user, and then discern which requests are sent from user. After that, files in ‘Service’ directory handles analyzed requests according to a set routine. If SQL query requires, then files in ‘Dao’ directory create SQL query and send to database. Files in ‘Dto’ create and maintain data objects.
    \item Server/src/main/java/com/dionysos/winecellar/domain/winecellar\\
    
This directory is a collection of files to handle user requests about winecellar. First, files from ‘Api’ directory receives the requests from user, and then discern which requests are sent from user. After that, files in ‘Service’ directory handles analyzed requests according to a set routine. If SQL query requires, then files in ‘Dao’ directory create SQL query and send to database. Files in ‘Dto’ create and maintain data objects.
\end{enumerate}
    \subsubsection{\textbf{Client}}
    \begin {enumerate}
    \item Welcome
    \begin{enumerate}
    \item Welcome\\
 This is the loading page of our application. When user clicks the image of wine glass, user can go to Kakao account-log in page. After log in, user encounters two red buttons, which is consisted of wine cellar registration and go to my winecellar. By using Welcome navigator, user can link to each page.  
\item WelcomeNavigator 
 This is a navigation which enables transition among WelcomeScreen, LoginScreen, SelectScreen and Registration.  
\end{enumerate}
\item Log in Page
 \begin{enumerate}
 \item login\\
 This is the first page which user can log in. When user clicks ‘Log in with kakao’, user can log in by entering the username and password of his Kakao account. At the same time, the information of user is saved in the database.  
\end{enumerate}
\item Registration
 \begin{enumerate}
     \item Registration\\ 
     This is the wine cellar registration page. The input data of wine cellar registration is the serial number of wine cellar. There are two cases in the wine cellar registration process. If user enters the serial number which is in the database, the serial number and the model name of the wine cellar is shown. Else if, ‘The serial number doesn’t exit’ message alerts and it lets user to enter the serial number again. After wine cellar registration, user can connect the physical wine cellar and wine cellar in application. 
     \item RegistrationNavigator\\
     This is a navigation which enables transition between SelectScreen and Registration.
 \end{enumerate}
 \begin{enumerate}
\item My winecellar\\
When user clicks the setting image, user goes to WineCellarSetting page. 
When user clicks the share image, user can share the image of wine cellar in Instagram. And ‘\#LGwinecellar, \#MyWineCellar, \#DIOnyoS’ hashtags are uploaded together in Instagram posting.
\item WineCellarSetting\\
 There are three parts in wine cellar setting page. First part is  nickname setting page. User can freely make the nickname of wine cellar. Second part is temperature control page. User can set the temperature of physical wine cellar by our application. Third part is Humidity control page. User can set the humidity of physical wine cellar by our application.
 \item WineTab\\
  It is tab which locates in the bottom of page. By clicking wine tab, user can move to wine cellar, lock main, wine registration, wine recommendation and share navigator. 
  \item WineCellarNavigator\\
This is a navigation which enables transition between WineCellar and WineCellarSetting.
\end{enumerate}
\item lock
\begin{enumerate}
    \item lock info\\
     This is a page for setting wine cellar password, which is for locking wine cellar. The password should be 6-digit password. When user saves the password, message which notifies that password is saved pops up.
    \item cellarlock\\
    
 This is a page which user can lock full floor or lock floor-by-floor. Locking full floor is done by clicking red button. And locking floor-by-floor is done by clicking each lock button. 
 \item locknavigator\\
 
 This is a navigation which enables transition among lock info, cellarlock and locknavigation.
\end{enumerate}
\item share
\begin{enumerate}
    \item sharehome\\
    This is a page which is consisted of three buttons. The first button is to link users to SharemyWineCellar. The second button is to link users to SharemyWineHistory. The third button is to link users to MakemyWineTopsterMain. 
    \item MyWineTopster\\
    This is a first page for explaining steps of making wine topster. When user reads all the direction and explanation, user clicks the go button, which links to MyWine.
    \item MyWineTopsterFirst\\
    This is a page for making wine topster. User can set the rows and columns of wine topster. Also, he can select background image from ‘Photos’ in his device. 
    \item MyWineTopsterNavigator\\
    This is a navigation which enables transition among sharehome, MyWineTopster and MyWineTopsterFirst.
\end{enumerate}





 This is the table which shows directory, file name and module name. 
 
 %%% 서버 파일들부터 수정 봐야함!!!
 
\begin{table}[ht!] \renewcommand\arraystretch{1.25}
\caption{Server Directory}
\begin{center}

\begin{tabular}{|p{4.3cm}|p{2.95cm}|p{1.3cm}|}
  \hline
 \textbf{Directory}& \textbf{File name} & \textbf{Module name} \\
\hline
  \textbf{Server/gradle/wrapper} & gradlewrapper.jar
gradlewrapper.properites
 & gradle\\
  \hline
  \textbf{Server/src/main/java/com/
  dionysos/winecellar/config} & WebConfig.java
 & Spring \\ 
  \hline
  \textbf{Server/src/main/java/com/
  dionysos/winecellar/domain} \par \textbf{\auth} & AuthController.java
  
  AuthResponse \par -Dto.java \par
  LoginDto.java

 & Spring\\
  \hline
 \textbf{Server/src/main/java/com/
 dionysos/winecellar/domain/} \par \textbf {member/}& LoginMemberId.java

MemberRepository.java

Member.java

MemberService.java \par
…

& Spring,
Lombok
 \\
\hline
  \textbf{/Server/src/main/java/com/
  dionysos/winecellar/domain} \par \textbf{wine/} & WineController.java
WineRepository.java
WineService.java
Wine.java

 & RSpring,
Lombok
 \\
  \hline
  \textbf{Server/src/main/java/com/
  dionysos/winecellar/domain} \par \textbf{/winecellar/} & Winecellar \par -Controller.java
  \par Winecellar \par -Repository.java

Winecellar.java

WineDto.java \par
…

 & Spring,
Lombok
\\
  \hline
\end{tabular}
\end{center}
\end{table}


\begin{table}[ht!] \renewcommand\arraystretch{1.25}
\caption{Client Directory}
    \begin{center}
        \begin{tabular}{|p{2.7cm}|p{3.4cm}|p{1.3cm}|}
        \hline
        \textbf{Directory}& \textbf{File name} & \textbf{Module name} \\
        
        \hline
        \textbf{src/screens/} \par \textbf{welcome} & Welcome.js
        
        WelcomeNavigator.js & React \par Native\\
        \hline
        
        \textbf{src/screens/} \par \textbf{welcome/login} & Login.js
        
        SocialWebview.js
        
        SocialWebviewModal.js & React \par Native
        
        Axios\\
        \hline
        
        \textbf{src/screens/} \par \textbf{winecellar} & Winecellar.js
         
        WineCellarNavigator.js
         
        WineCellarSetting.js 
        
        WineTab.js & Spring\\
        \hline
        
        \textbf{src/screens/} \par \textbf{registration} & Registration.js
        
        RegistrationNavigator.js & React \par Native\\
        \hline
         
        \textbf{src/screens/share} & ShareHome.js
        
        ShareNavigator.js
        
        ShareWineCellar.js 
        
        InstagramShare.js
        
        MyWineTopsterNavigator.js
        
        MyWineTopsterMain.js
        
        MyWineTopsterFirst
        
        MyWineTopsterSecond
        
        MyWineTopsterThird & React \par Native\\
        \hline
        
        \textbf{src/screens/lock} & LockMain.js
        
        LockNavigator.js
        
        LockInfo.js
        
        CellarLock.js 
        
        LockGate.js & React \par Native\\
        \hline
        
        \textbf{src/context} & WinecellarContext.js & React\\
        \hline
        
        \textbf{src/api} & winecellarApi.js & React \par Native\\
        \hline
        
        \end{tabular}
    \end{center}
\end{table}

    \subsection{Modules}
    \begin{enumerate}
        \item React Native\\
        We use React Native to develop a wine cellar management application for the Apple device, such as IPhone and IPad. It acts as a connection between the user and the server. It also receives the user’s input value, passes it to the backend and shows the stored value of database. 
        \item Axios\\
        We use Axios module for handling request and response from REST API, such as acquiring information of one's wine cellar and sending the password of wine cellar to enable/disable cellar lock function. Axios is Promise based HTTP client for the browser and Node.js. It is used to request data from API server.
        \item React\\
        React Native is based on React. 
        React is a free and open-source front-end JavaScript library for building user interfaces based on UI components. 
    \end{enumerate}
    
\section{Use cases}
    \subsection{Frontend UI}
    \begin{enumerate}
        \item WineCellar Main Page
        \begin{enumerate}
            \item My WineCellar Page\\
            This is a main page of DIOnyoS. My WineCellar Page shows statement of the physical LG winecellar. Depending on the winecellar model, the number of floors varies. The page shows the images of wine label saved in the application. The title container has two icons, which are Setting and Share. By pressing Setting icon, user goes to the My WineCellar setting page. And By pressing Share icon, user can share the image of winecellar in the Instagram with the hashtag '\#LGwinecellar \#MyWineCellar \#DIOnyoS'.    
            \item WineCellar Tab Bar\\
            At the bottom of the page, there are tab navigation bar which is consisted of three icons, WineCellar, Lock and Share. 
            \begin{enumerate}
                \item WineCellar
                \item Lock\\
                Lock page is the one of main functions of the application. 
                \item Share
            \end{enumerate}
            \item Lock
            \begin{enumerate}
                \item Password: Setting\\
                \begin{figure}
                    \centering
                    \includegraphics[width=4cm, height=6cm]{2. WineCellarPasswdSetting.png}
                    \caption{Wine Cellar Password Setting}
                \end{figure}
                Before entering Lock Page, user must set 6-digit password of the wine cellar. The password is encrypted and saved in the server's database. 
                 
                \item Password: Saved\\
                \begin{figure}
                    \centering          
                    \includegraphics[width=4cm, height=6cm]{3. Winecellar AfterSetPasswd.png}
                    \caption{Wine Cellar Password Check}
                \end{figure}
                After pressing Save button, pop-up window appears. 
                 
                \item Wine Cellar Main Page : Enter Password
                \begin{figure}
                    \centering
                    \includegraphics[width=4cm, height=6cm]{4. Passwd Confirmation.png}
                    \caption{Wine Cellar Lock Screen}
                \end{figure}
                
                \item WineCellar Lock: Main Page
                \begin{figure}
                    \centering
                    \includegraphics[width=4cm, height=6cm]{5. ViewingCellarTrytoLock.png}
                    \caption{Wine Cellar Lock Settings}
                \end{figure}
                \begin{figure}
                    \centering
                    \includegraphics[width=4cm, height=6cm]{6.PartiallyLoced.png}
                    \caption{Successful Wine Cellar Lock}
                \end{figure}
                 
                First, user can lock the full floor of the wine cellar by pressing the red button. Second, they can lock the floor-by-floor by pressing each lock icon.  
                \item Successful Lock\\
                When lock is successful, pop-up window appears. 
            \end{enumerate}
        \end{enumerate}
    \end{enumerate}
\end{enumerate}


\end{document}
\endinput
